\chapter{Structure}
It is a good idea to discuss the structure of the report with your supervisor rather early in your writing. Given next is a generic structure that is a starting point, but by no means the absolute standard. Your supervisor should provide a better structure for the specific field you are writing your thesis in. Note also that the naming of the chapters is not compulsory, but may be a helpful guideline.
\begin{description}
\item[Introduction] should give the background of your work. Important parts to cover:
\begin{itemize}
\item Give the context of your work, have a short introduction to the area.
\item Define the problem you are solving (or trying to solve).
\item Specify your contributions. What does this particular work/report bring to the research are or to the body of knowledge? How is the work divided between the co-authors? (This part is essential to pinpoint individual work. For theses with two authors, it is compulsory to identify which author has contributed with which part, both with respect to the work and the report.)
\item Describe related work (literature study). Besides listing other work in the area, mention how is it related or relevant to your work. The tradition in some research area is to place this part at the end of the report (check with your supervisor).
\end{itemize}
\item[Approach] should contain a description of your solution(s), with all the theoretical background needed. On occasion this is replaced by a subset or all of the following:
\begin{itemize}
\item \textbf{Method}: describe how you go about solving the problem you defined. Also how do you show/prove that your solution actually works, and how well does it work.
\item \textbf{Theory}: should contain the theoretical background needed to understand your work, if necessary.
\item \textbf{Implementation}: if your work involved building an artefact/implementation, give the details here. Note, that this should not, as a rule, be a chronological description of your efforts, but a view of the result. There is a place for insights and lamentation later on in the report, in the Discussion section.
\end{itemize}
\item[Evaluation] is the part where you present the finds. Depending on the area this part contains a subset or all of the following: 
\begin{itemize}
\item \textbf{Experimental Setup} should describe the details of the method used to evaluate your solution(s)/approach. Sometimes this is already addressed in the \textbf{Method}, sometimes this part replaces \textbf{Method}.
\item \textbf{Results} contains the data (as tables, graphs) obtained via experiments  (benchmarking, polls, interviews).
\item \textbf{Discussion} allows for a longer discussion and interpretation of the results from the evaluation, including extrapolations and/or expected impact. This might also be a good place to describe your positive and negative experiences related to the work you carried out.
\end{itemize} 
Occasionally these sections are intermingled, if this allows for a better presentation of your work. However, try to distinguish between measurements or hard data (results) and extrapolations, interpretations, or speculations (discussion).
\item[Conclusions] should summarize your findings and possible improvements or recommendations.
\item[Bibliography] is a must in a scientific report. {\LaTeX} and \texttt{bibtex} offer great support for  handling references and automatically generating bibliographies.
\item[Appendices] should contain lengthy details of the experimental setup, mathematical proofs, code download information, and shorter code snippets. Avoid longer code listings. Source code should rather be made available for download on a website or on-line repository of your choosing.

\end{description}

